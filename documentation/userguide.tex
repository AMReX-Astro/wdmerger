\documentclass[12pt]{book} 

\usepackage[margin=1.0in]{geometry}

\usepackage{amsmath}
\usepackage{amssymb}

\usepackage{epsfig}

\usepackage{natbib}

% chapter title styles
\usepackage[Sonny]{fncychap}
\ChNameVar{\LARGE}
\ChTitleVar{\LARGE\sl}

% part page style see
% http://tex.stackexchange.com/questions/6609/problems-with-part-labels-using-titlesec
\usepackage{titlesec}

\titleformat{\part}[display]
   {\Huge\filcenter}
   {{\partname{}} \thepart}
   {0em}
   {\hrule}


% hyperlinks -- load after fncychap
\usepackage{hyperref}

% color package
\usepackage[usenames]{color}

% number subsubsections and put them in the TOC
\setcounter{tocdepth}{3}
\setcounter{secnumdepth}{3}

% custom hrule for title page
\newcommand{\HRule}{\rule{\linewidth}{0.125mm}}


% short table of contents
\usepackage{shorttoc}

% spacing in the table of contents
\usepackage[titles]{tocloft}

\setlength{\cftbeforechapskip}{2ex}
\setlength{\cftbeforesecskip}{0.25ex}

% don't put a header on blank pages, see
% http://www.latex-community.org/forum/viewtopic.php?f=4&p=51559
% change ``plain'' to ``empty'' to eliminate the page number
\makeatletter
\renewcommand*\cleardoublepage{\clearpage\if@twoside
\ifodd\c@page\else
\hbox{}
\thispagestyle{empty}
\newpage
\if@twocolumn\hbox{}\newpage\fi\fi\fi}
\makeatother


% don't make the chapter/section headings uppercase.  See the fancyhdr
% documentation (section 9)
\usepackage{fancyhdr}
\renewcommand{\chaptermark}[1]{%
 \markboth{\chaptername
\ \thechapter.\ #1}{}}

\renewcommand{\sectionmark}[1]{\markright{\thesection---#1}}


% skip a bit of space between paragraphs, to enhance readability
\usepackage{parskip}



% special fraction
\newcommand{\sfrac}[2]{\mathchoice
  {\kern0em\raise.5ex\hbox{\the\scriptfont0 #1}\kern-.15em/
   \kern-.15em\lower.25ex\hbox{\the\scriptfont0 #2}}
  {\kern0em\raise.5ex\hbox{\the\scriptfont0 #1}\kern-.15em/
   \kern-.15em\lower.25ex\hbox{\the\scriptfont0 #2}}
  {\kern0em\raise.5ex\hbox{\the\scriptscriptfont0 #1}\kern-.2em/
   \kern-.15em\lower.25ex\hbox{\the\scriptscriptfont0 #2}}
  {#1\!/#2}}

% codes
\newcommand{\castro}{{\sf Castro}}
\newcommand{\boxlib}{{\sf BoxLib}}
\newcommand{\yt}{{\sf yt}}


%------------------------------------------------------------------------------
\begin{document}

\frontmatter

\begin{titlepage}
\begin{center}
\ \\[3in]
{\sf \Huge WDMERGER} 

\begin{minipage}{5.5in}
\HRule\\[2mm]
\centering
{\Large \em A software package for simulating white dwarf mergers with CASTRO}

\HRule
\end{minipage}

\ \\[1 in]
{\sf \huge User's Guide}

\vfill

{\large \today}
\end{center}

\end{titlepage}


\shorttoc{Chapter Listing}{0}

\setcounter{tocdepth}{2}
\tableofcontents

\clearpage

\addcontentsline{toc}{chapter}{preface}

\clearpage

\mainmatter

\chapter{Getting Started}

Welcome to the user's guide for the wdmerger software package. This software is designed to
simulate binary white dwarf systems, and is intended to provide useful information on the 
viability of mergers of white dwarfs as a progenitor for Type Ia supernovae. It is a set of 
source files designed to use the CASTRO code \citep{castro}.

This software requires a few other software packages to be installed before it can be used.
The science software includes \href{https://ccse.lbl.gov/BoxLib/}{BoxLib} and 
\href{https://ccse.lbl.gov/Downloads/downloadCASTRO.html}{CASTRO}. This software is written 
in C++ and Fortran, and compilers for each are needed. The gcc compilers are sufficient.
In addition, it relies on several utilities that are 
commonly distributed on Linux systems: \texttt{bash}, \texttt{GNU bc}, \texttt{GNU make}, 
\texttt{batch}, \texttt{GNU sed}, and \texttt{GNU awk}. Analysis is mainly performed in 
\href{https://www.python.org/}{python}, with the \href{http://matplotlib.org/}{matplotlib} 
library and the \href{http://yt-project.org/}{yt} code being used for visualization. 
If python and yt are not installed on your system, you will still be able to compile and 
run this code, but you will need some other mechanism for analyzing the BoxLib output data.
We have some routines for this purpose available in Fortran and may be able to provide them
upon request.

%------------------------------------------------------------------------------
\backmatter

\bibliographystyle{../papers/apj}
\bibliography{../papers/refs}

\end{document}
