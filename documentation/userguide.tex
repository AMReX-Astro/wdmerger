\documentclass[12pt]{book} 

\usepackage[margin=1.0in]{geometry}

\usepackage{amsmath}
\usepackage{amssymb}

\usepackage{epsfig}

\usepackage{natbib}

% chapter title styles
\usepackage[Sonny]{fncychap}
\ChNameVar{\LARGE}
\ChTitleVar{\LARGE\sl}

% part page style see
% http://tex.stackexchange.com/questions/6609/problems-with-part-labels-using-titlesec
\usepackage{titlesec}

\titleformat{\part}[display]
   {\Huge\filcenter}
   {{\partname{}} \thepart}
   {0em}
   {\hrule}


% hyperlinks -- load after fncychap
\usepackage{hyperref}

% color package
\usepackage[usenames]{color}

% number subsubsections and put them in the TOC
\setcounter{tocdepth}{3}
\setcounter{secnumdepth}{3}

% custom hrule for title page
\newcommand{\HRule}{\rule{\linewidth}{0.125mm}}


% short table of contents
\usepackage{shorttoc}

% spacing in the table of contents
\usepackage[titles]{tocloft}

\setlength{\cftbeforechapskip}{2ex}
\setlength{\cftbeforesecskip}{0.25ex}

% don't put a header on blank pages, see
% http://www.latex-community.org/forum/viewtopic.php?f=4&p=51559
% change ``plain'' to ``empty'' to eliminate the page number
\makeatletter
\renewcommand*\cleardoublepage{\clearpage\if@twoside
\ifodd\c@page\else
\hbox{}
\thispagestyle{empty}
\newpage
\if@twocolumn\hbox{}\newpage\fi\fi\fi}
\makeatother


% don't make the chapter/section headings uppercase.  See the fancyhdr
% documentation (section 9)
\usepackage{fancyhdr}
\renewcommand{\chaptermark}[1]{%
 \markboth{\chaptername
\ \thechapter.\ #1}{}}

\renewcommand{\sectionmark}[1]{\markright{\thesection---#1}}


% skip a bit of space between paragraphs, to enhance readability
\usepackage{parskip}



% special fraction
\newcommand{\sfrac}[2]{\mathchoice
  {\kern0em\raise.5ex\hbox{\the\scriptfont0 #1}\kern-.15em/
   \kern-.15em\lower.25ex\hbox{\the\scriptfont0 #2}}
  {\kern0em\raise.5ex\hbox{\the\scriptfont0 #1}\kern-.15em/
   \kern-.15em\lower.25ex\hbox{\the\scriptfont0 #2}}
  {\kern0em\raise.5ex\hbox{\the\scriptscriptfont0 #1}\kern-.2em/
   \kern-.15em\lower.25ex\hbox{\the\scriptscriptfont0 #2}}
  {#1\!/#2}}

% codes
\newcommand{\castro}{{\sf Castro}}
\newcommand{\boxlib}{{\sf BoxLib}}
\newcommand{\yt}{{\sf yt}}


%------------------------------------------------------------------------------
\begin{document}

\frontmatter

\begin{titlepage}
\begin{center}
\ \\[3in]
{\sf \Huge WDMERGER} 

\begin{minipage}{5.5in}
\HRule\\[2mm]
\centering
{\Large \em A software package for simulating white dwarf mergers with CASTRO}

\HRule
\end{minipage}

\ \\[1 in]
{\sf \huge User's Guide}

\vfill

{\large \today}
\end{center}

\end{titlepage}


\shorttoc{Chapter Listing}{0}

\setcounter{tocdepth}{2}
\tableofcontents

\clearpage

\addcontentsline{toc}{chapter}{preface}

\clearpage

\mainmatter

\chapter{Getting Started}

Welcome to the user's guide for the wdmerger software package. This software is designed to
simulate binary white dwarf systems, and is intended to provide useful information on the 
viability of mergers of white dwarfs as a progenitor for Type Ia supernovae. It is a set of 
source files designed to use the CASTRO code \citep{castro}.

This software requires a few other software packages to be installed before it can be used.
The science software includes \href{https://ccse.lbl.gov/BoxLib/}{BoxLib} and 
\href{https://ccse.lbl.gov/Downloads/downloadCASTRO.html}{CASTRO}. This software is written 
in C++ and Fortran, and compilers for each are needed. The gcc compilers are sufficient.
In addition, it relies on several utilities that are 
commonly distributed on Linux systems: \texttt{bash}, \texttt{GNU bc}, \texttt{GNU make}, 
\texttt{batch}, \texttt{GNU sed}, and \texttt{GNU awk}. Analysis is mainly performed in 
\href{https://www.python.org/}{python}, with the \href{http://matplotlib.org/}{matplotlib} 
library and the \href{http://yt-project.org/}{yt} code being used for visualization. 
If python and yt are not installed on your system, you will still be able to compile and 
run this code, but you will need some other mechanism for analyzing the BoxLib output data.
We have some routines for this purpose available in Fortran and may be able to provide them
upon request.

This software assumes that you have set several environment variables:
\begin{itemize}
  \item \texttt{BOXLIB\_HOME}, the location of the BoxLib code's top-level directory
  \item \texttt{CASTRO\_DIR}, the location of the CASTRO code's top-level directory
  \item \texttt{WDMERGER\_HOME}, the location of this software's top-level directory
\end{itemize}

In addition, you should include in your \texttt{PYTHONPATH} the location of the 
\texttt{analysis} subdirectory. In \texttt{bash}, include the following in your 
\texttt{.bashrc} profile file:

\texttt{export PYTHONPATH=\$PYTHONPATH:\$WDMERGER\_HOME/analysis}

\chapter{Code Structure}

There are several top-level directories in the \texttt{wdmerger} project. Here we will describe
the contents of each. There are also a \texttt{LICENSE.txt} file containing the software license,
and a \texttt{README.md} file describing how to get started.

The \texttt{documentation} directory contains this user's guide and is intended to provide information
on how to use and run this software.

The \texttt{source} directory contains all of the Fortran routines needed to build a CASTRO 
executable file. It is not normally built directly, but you can type \texttt{make} in that directory
to build a vanilla executable built with the default settings.

The \texttt{analysis} directory contains the Python analysis scripts used for analyzing output
data (both in BoxLib and plaintext format). 

The \texttt{tests} directory includes problem setups that are used for testing 
the hydrodynamics code CASTRO and exercises a wide range of its capabilities.

The \texttt{job\_scripts} directory contains utilities that are used for building and running
problem setups on various machine architectures.

The \texttt{papers} directory includes the LaTeX source code for the journal articles
that are being submitted as a result of this project.

\chapter{Test Problems}

The \texttt{tests} directory contains a number of example problems that we use
to assess the reliability of CASTRO in performing hydrodynamics simulations 
with self-gravity. This section includes a list of the existing problems 
as well as the instructions for building and running a problem.

\begin{itemize}
  \item The \texttt{gravity} subdirectory includes problems that test the Poisson 
self-gravity module in CASTRO.
  \begin{itemize}
    \item The \texttt{boundary\_condition\_comparison} test checks the validity of our 
multipole approximation for the boundary conditions on the Poisson equation. It runs
the gravity solve for several different choices of the number of expansion coefficients
and compares the resulting gravitational potential to the exact result.
    \item The \texttt{uniform\_cube\_sphere} test is designed to check the convergence
properties of the Poisson solver. It loads either a sphere or cube of uniform density
onto the computational domain and then calculates the potential at several different
spatial resolutions to check if the potential converges.
  \end{itemize}
  \item The \texttt{hydro} subdirectory tests includes tests that use the hydrodynamics 
module, including many with self-gravity.
  \begin{itemize}
    \item The \texttt{circular\_orbit} test loads two white dwarfs on a grid and then orbits
them for a large number of periods, to test conservation of linear momentum,
angular momentum and energy convergence.
    \item The \texttt{evrard\_collapse} test runs the test problem described originally by 
\cite{evrard:1988} and whose results for a grid-based code can be seen in \cite{arepo}. A spherical
distribution of gas obeying a gamma-law equation of state is loaded onto a grid with negligible 
pressure. The gas collapses, rebounds with a shock, and then settles into hydrodynamic equilibrium.
It is a good test of the energy conservation properties of the code.
    \item The \texttt{freefall} test loads two white dwarfs onto a grid and allows them to fall towards
each other under their mutual gravitational influence. We check to see whether the timescale for them
colliding matches the analytical expectation for two point masses.
    \item The \texttt{kelvin\_helmoltz} test runs several varieties of a 2D shearing fluid that exhibits the 
Kelvin-Helmholtz instability. The relevant papers are \cite{robertson:2010}, \cite{arepo}, and \cite{mcnally:2012}.
    \item The \texttt{single\_star\_hse} test loads a single white dwarf on a grid and checks how well 
hydrostatic equilibrium is maintained.
  \end{itemize}
  \item The \texttt{performance} subdirectory examines the scaling and performance properties of CASTRO.
  \begin{itemize}
    \item The \texttt{strong\_scaling} test runs the same problem with a varying number of processors and 
checks the scaling properties.
    \item The \texttt{weak\_scaling} test scales the size of the problem with the number of processors and
checks the scaling properties.
  \end{itemize}
\end{itemize}

Each test directory is set up with a uniform pattern. The \texttt{source/} subdirectory includes all of the source code
needed to build the test. The \texttt{makefile} builds the executable and creates
the inputs files needed for the simulation. The \texttt{run\_test.sh} script executes the test by creating a subdirectory
for each parameter set and then submitting it with a batch submission script (for a vanilla Linux installation,
this will use the \texttt{batch} utility; for a limited number of clusters and supercomputers, 
there will be recommended options and specialized submission scripts). The \texttt{analysis.py} Python script
is used to analyze the output and generature figures. In summary, to run a test, perform the following series of commands:
\begin{itemize}
  \item \texttt{make}
  \item \texttt{bash run\_test.sh}
  \item \texttt{python analysis.py} (after the execution has completed)
\end{itemize}

%------------------------------------------------------------------------------
\backmatter

\bibliographystyle{../papers/apj}
\bibliography{../papers/refs}

\end{document}
