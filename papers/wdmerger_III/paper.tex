\documentclass[twocolumn,numberedappendix]{../aastex6}

% these lines seem necessary for pdflatex to get the paper size right
\pdfpagewidth 8.5in
\pdfpageheight 11.0in

% for the red MarginPars
\usepackage{color}

% some extra math symbols
\usepackage{mathtools}

% allows Greek symbols to be bold
\usepackage{bm}

% allows us to force the location of a figure
\usepackage{float}

% allows comment sections
\usepackage{verbatim}

% Override choices in \autoref
\def\sectionautorefname{Section}
\def\subsectionautorefname{Section}
\def\subsubsectionautorefname{Section}

% MarginPars
\setlength{\marginparwidth}{0.75in}
\newcommand{\MarginPar}[1]{\marginpar{\vskip-\baselineskip\raggedright\tiny\sffamily\hrule\smallskip{\color{red}#1}\par\smallskip\hrule}}

\newcommand{\msolar}{\mathrm{M}_\odot}

% Software names
\newcommand{\boxlib}{\texttt{BoxLib}}
\newcommand{\castro}{\texttt{CASTRO}}
\newcommand{\wdmerger}{\texttt{wdmerger}}
\newcommand{\python}{\texttt{Python}}
\newcommand{\matplotlib}{\texttt{matplotlib}}
\newcommand{\yt}{\texttt{yt}}

\begin{document}

%==========================================================================
% Title
%==========================================================================
\title{White Dwarf Mergers on Adaptive Meshes\\ III. Mergers}

\shorttitle{WD Mergers. III. Mergers}
\shortauthors{Katz et al. (2016)}

\author{TBD}
%==========================================================================
% Abstract
%==========================================================================
\begin{abstract}
We describe a method for constructing equilibrium initial conditions for a binary star system,
for a general equation of state. We extend this method to include the effects of adaptive
mesh refinement in the algorithm. Then we consider mergers of these binary stars.

\end{abstract}
\keywords{hydrodynamics - methods: numerical - supernovae: general - white dwarfs}

%==========================================================================
% Introduction
%==========================================================================
\section{Introduction}




%==========================================================================
% Numerical Methods
%==========================================================================
\section{Numerical Methods}

\subsection{Hybrid Advection Scheme}

It is well known that there is a trade-off between conservation of linear
momentum and conservation of angular momentum in grid-based, Eulerian codes
such as \castro. Although both are true statements analytically, we cannot
numerically guarantee that both hold. Consequently the choice of which
physical quantity to conserve must be motivated by analysis of the particular
problem to be done. Ideally a simulation code will be capable of evolving the
equations for both the linear momentum and the angular momentum, and be
able to switch between them as appropriate. Prior to this work, \castro\
already had the standard method implemented for evolving the linear momentum.
To solve the need for an angular momentum evolution, we adopt the ``hybrid''
advection scheme presented by \cite{byerly:2014} (see also \cite{motl:2002} for
a similar equation set). Their approach is appropriate for physical situations
where there is a dominant angular momentum axis and it is desirable to conserve
that component as accurately as possible. This is certainly applicable for
studies of the stability of binary stellar systems. This section contains a
discussion of our implementation of that scheme in \castro. For the sake
of simplicity, let us assume that the rotation axis of the binary system is the $z$
axis and so the WDs orbit in the $xy$-plane. The core of the method is that instead
of solving the Euler equations for the linear momenta $(\rho u)$ and $(\rho v)$,
we instead solve the corresponding equations for the radial momentum
$s_R \equiv \rho v_R$ (where $v_R$ is the radial velocity with respect to the
rotation axis) and the angular momentum with respect to the $z$-axis,
$\ell_z \equiv R\rho v_\phi$ (where $v_\phi$ is the azimuthal velocity).
The equation for the linear $z$-momentum is unchanged. We want to be able
to solve these equations while still operating in our preferred Cartesian
coordinate system. \cite{byerly:2014} present these equations as the following;
for the moment we neglect inclusion of external source terms such as gravity and rotation:
\begin{align}
  \frac{\partial(s_R)}{\partial t} + \nabla \cdot (s_R \mathbf{u}) &=
    - \frac{1}{R}\left(x \frac{\partial}{\partial x} + y \frac{\partial}{\partial y}\right) p
    + \frac{\ell_z^2}{\rho R^3} \label{eq:radial-momentum}\\
    \frac{\partial(\ell_z)}{\partial t} + \nabla \cdot (\ell_z \mathbf{u}) &=
  \left(y\frac{\partial}{\partial x} - x \frac{\partial}{\partial y}\right) p \label{eq:angular-momentum}
\end{align}  
Here $\mathbf{u}$ is the ordinary linear momentum used to advect any fluid quantity
on the grid, and $R = \sqrt{x^2 + y^2}$, where the coordinates $x$ and $y$ are defined
relative to the axis of rotation, and the origin is located on a zone corner
so that singularities are avoided. Note that $v_R = (1 / R)(x u + y v)$, and
$v_\phi = (1/R)(x v - y u)$. These equations can be straightforwardly derived from
the Euler equations for linear momentum by appropriate multiplication of $x$, $y$,
$x / R$, and $y/R$, and subsequent algebraic manipulation. We now rewrite this in the
way that illustrates how the update is done in the code:
\begin{align}
  \frac{\partial(s_R)}{\partial t} &= -\nabla \cdot (s_R \mathbf{u})
    - \frac{1}{R}\left(x \frac{\partial p}{\partial x} + y \frac{\partial p}{\partial y}\right)
    + \frac{\ell_z^2}{\rho R^3} \label{eq:radial-momentum-rearranged}\\
    \frac{\partial(\ell_z)}{\partial t} &= -\nabla \cdot (\ell_z \mathbf{u}) - \left(\frac{\partial (-py)}{\partial x} + \frac{\partial(px)}{\partial y}\right).\label{eq:angular-momentum-rearranged}
\end{align}
The first term is the standard advective flux term for any variable that advects with
the flow, and the state quantity inside this advective term can be derived on cell edges
using the primitive hydrodynamic variables after a Riemann solver has been performed.
The second term on the right-hand-side of \autoref{eq:angular-momentum-rearranged} can be swept up
into the flux terms in the same way that the pressure term for the linear momenta
is often added to the fluxes, to make the conservation form explicit. The remainder
are the last two terms on the right-hand-side of \autoref{eq:radial-momentum-rearranged}.
While maintaining second-order accuracy in time, the last term can be treated as a
cell-centered source term that we deal with using a predictor-corrector
method (similar to how we implement external forcing like gravity and rotation; see
\cite{castro} for details). The pressure gradients in the second-to-last term can be
constructed using the Riemann-solved edge states, and they are multiplied by zone
coordinates that are cell-centered. In principle this is second-order accurate in time
because the edge states are too; however, as a drawback, the pressure is unaware of the
source terms that are applied in the corrector step after a hydrodynamics update. The
benefit of this approach is that we do not need to construct a general cell-centered
gradient of the pressure using cell-centered state data, nor do we need to perform the
equation of state call that entails.

It is important to observe that in this formulation the radial momentum equation
is \textit{not} a conserved quantity: it should and will in general change with
time over the course of the simulation if the system is not in perfect rotational
equilibrium. The above form, absent of source terms, makes this clear. The angular momentum term in
\autoref{eq:radial-momentum} can be thought of physically as expressing the fact
that a particle with some initial angular momentum will have its radial momentum
increase with time as it moves away from the origin, and can be thought of geometrically
as expressing the curved nature of the underlying coordinates for the radial momentum
\citep{motl:2002}. It is only the presence of a source term like gravity that can keep
such a particle on an orbit at its original radius, and for such a case it may be helpful
to think of the angular momentum source term as an outward centrifugal force that
balances against the inward centripetal force provided by gravity.

In the absence of source terms, the angular momentum \textit{is} conserved
to machine precision (ignoring the effects of physical domain boundaries). In the
presence of source terms, the error in angular momentum conservation is of a
similar order of magnitude to the error in linear momentum conservation under
the influence of those source terms (see \autoref{eq:force-angular}). Since this
error is quite small for gravitational and rotational forces in practice, angular
momentum conservation is quite good, especially in comparison to the standard
method of evolving only the linear momentum. Thus this method is most appropriate
when the net radial momentum is small in comparison to the net angular momentum,
that is, when the motion is primarily azimuthal in nature.

To implement this method in \castro, we desired an approach that was flexible
and would leave as little imprint on the code structure as possible.
To this end, what we have done is to add three new ``hybrid'' momentum state
variables, corresponding to the radial momentum, angular momentum, and the linear
momentum component that is perpendicular to both. Whenever we update the normal
momentum state variables with hydrodynamics fluxes, we add the analogous fluxes
to the hybrid momenta -- that is, using the edge state values determined by the
final multi-dimensional Riemann solve, we construct the value of the hybrid
momenta on that zone edge, and then allow it to be transported as usual by the
advective velocity $\mathbf{u}$. When we update the normal momenta with an
external forcing $\mathbf{F}$ such as gravity or rotation, with $x$ and $y$
components $F_x$ and $F_y$, we apply the update to the hybrid momenta as well:
\begin{align}
  \left.\frac{\partial(s_R)}{\partial t}\right|_{F} &= -F_x \frac{x}{R} - F_y\frac{y}{R} \label{eq:force-radial}\\
  \left.\frac{\partial(\ell_z)}{\partial t}\right|_{F} &= F_x\, y - F_y\, x.\label{eq:force-angular}
  \end{align}
Then, when we are in a phase of the evolution when we want to conserve angular
momentum, we make one additional change: at the end of a (single-level) advance,
we recompute the normal momenta so that they are fully consistent with the hybrid
momenta. This is all that is necessary to have a calculation that conserves angular
momentum over the course of the simulation, without changing any of the
core infrastructure for the hydrodynamics update. For example, when we compute
the primitive variables from the conservative variables, we will still be
getting them from the normal momenta, but these normal momenta are always consistent
with the conserved angular momentum, so we get the same result as if we had done
a much more involved update where we explicitly computed the primitive variables
using the hybrid momenta. There is of course no unique choice for how to do this.
For example, we could take the approach of explicitly re-calculating the normal
momenta after any change has been made to the state. This would add significant code
complexity without a clear benefit. The most significant difference would be
that when evaluating the new-time value of source terms like rotation, the velocity
field used is slightly different than it would otherwise be; but this is a high-order
effect that does not change the core angular momentum conservation property. An
analysis of the different mechanisms for implementing this update may be an interesting
avenue for future research --- including various aspects like how to implement the
source terms in the radial momentum equation. Of course, the answer will likely
be problem-dependent. Indeed, we already face this situation in general with the
gravitational and rotational source terms, even independent of the hybrid momentum
discussion --- the velocity field seen by the new-time rotation source terms depends
on whether we apply the gravitational force before or after the rotation force.



%==========================================================================
% Conclusions
%==========================================================================
\section{Conclusions and Discussion}\label{Sec:Conclusions and Discussion}


\acknowledgments

This research was supported by NSF award AST-1211563.  This research
used resources of the National Energy Research Scientific Computing
Center, which is supported by the Office of Science of the
U.S. Department of Energy under Contract No. DE-AC02-05CH11231.  An
award of computer time was provided by the Innovative and Novel
Computational Impact on Theory and Experiment (INCITE) program.  This
research used resources of the Oak Ridge Leadership Computing Facility
located in the Oak Ridge National Laboratory, which is supported by
the Office of Science of the Department of Energy under Contract
DE-AC05-00OR22725.

\clearpage

\bibliographystyle{../aasjournal}
\bibliography{../refs}


\clearpage
\appendix



\end{document}

