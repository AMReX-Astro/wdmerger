\documentclass[iop]{../emulateapj}
% for \sout
\usepackage{ulem}
% makes sure \em{} is italic rather than underlined (corrects ulem from line above)
\normalem

% for the red MarginPars
\usepackage{color}

% some extra math symbols
\usepackage{mathtools}

\newcommand{\gcc}{\mathrm{g~cm^{-3} }}

% MarginPars
\setlength{\marginparwidth}{0.75in}
\newcommand{\MarginPar}[1]{\marginpar{\vskip-\baselineskip\raggedright\tiny\sffamily\hrule\smallskip{\color{red}#1}\par\smallskip\hrule}}


\newcommand{\evm}{{(-)}}
\newcommand{\evz}{{(\circ)}}
\newcommand{\evp}{{(+)}}
\newcommand{\enu}{{(\nu)}}



\newcommand{\msolar}{\mathrm{M}_\odot}

\begin{document}

%==========================================================================
% Title
%==========================================================================
\title{Double White Dwarf Mergers on Adaptive Meshes\\ III. Mergers}

\shorttitle{DWD Mergers. III. Mergers}
\shortauthors{Katz et al. (2016)}

\author{TBD}
%==========================================================================
% Abstract
%==========================================================================
\begin{abstract}
We describe a method for constructing equilibrium initial conditions for a binary star system,
for a general equation of state. We extend this method to include the effects of adaptive
mesh refinement in the algorithm. Then we consider mergers of these binary stars.

\end{abstract}
\keywords{hydrodynamics - methods: numerical - supernovae: general - white dwarfs}

%==========================================================================
% Introduction
%==========================================================================
\section{Introduction}

\cite{castro}


%==========================================================================
% Conclusions
%==========================================================================
\section{Conclusions and Discussion}\label{Sec:Conclusions and Discussion}


\acknowledgments

This research was supported by NSF award AST-1211563.  This research
used resources of the National Energy Research Scientific Computing
Center, which is supported by the Office of Science of the
U.S. Department of Energy under Contract No. DE-AC02-05CH11231.  An
award of computer time was provided by the Innovative and Novel
Computational Impact on Theory and Experiment (INCITE) program.  This
research used resources of the Oak Ridge Leadership Computing Facility
located in the Oak Ridge National Laboratory, which is supported by
the Office of Science of the Department of Energy under Contract
DE-AC05-00OR22725.

\clearpage

\bibliographystyle{../apj}
\bibliography{../refs}


\clearpage
\appendix



\end{document}

