\documentclass[twocolumn,numberedappendix]{../aastex6}

% these lines seem necessary for pdflatex to get the paper size right
\pdfpagewidth 8.5in
\pdfpageheight 11.0in

% for the red MarginPars
\usepackage{color}

% some extra math symbols
\usepackage{mathtools}

% allows Greek symbols to be bold
\usepackage{bm}

% allows us to force the location of a figure
\usepackage{float}

% allows comment sections
\usepackage{verbatim}

% Override choices in \autoref
\def\sectionautorefname{Section}
\def\subsectionautorefname{Section}
\def\subsubsectionautorefname{Section}

% MarginPars
\setlength{\marginparwidth}{0.75in}
\newcommand{\MarginPar}[1]{\marginpar{\vskip-\baselineskip\raggedright\tiny\sffamily\hrule\smallskip{\color{red}#1}\par\smallskip\hrule}}

\newcommand{\msolar}{\mathrm{M}_\odot}

% Software names
\newcommand{\boxlib}{\texttt{BoxLib}}
\newcommand{\castro}{\texttt{CASTRO}}
\newcommand{\wdmerger}{\texttt{wdmerger}}
\newcommand{\python}{\texttt{Python}}
\newcommand{\matplotlib}{\texttt{matplotlib}}
\newcommand{\yt}{\texttt{yt}}

\begin{document}

%==========================================================================
% Title
%==========================================================================
\title{White Dwarf Mergers on Adaptive Meshes\\ III. Mergers}

\shorttitle{WD Mergers. III. Mergers}
\shortauthors{Katz et al. (2016)}

\author{TBD}
%==========================================================================
% Abstract
%==========================================================================
\begin{abstract}
We describe a method for constructing equilibrium initial conditions for a binary star system,
for a general equation of state. We extend this method to include the effects of adaptive
mesh refinement in the algorithm. Then we consider mergers of these binary stars.

\end{abstract}
\keywords{hydrodynamics - methods: numerical - supernovae: general - white dwarfs}

%==========================================================================
% Introduction
%==========================================================================
\section{Introduction}




%==========================================================================
% Numerical Methods
%==========================================================================
\section{Numerical Methods}
\label{sec:methodology}

The numerical methodology for our problem was established in \cite{wdmergerI}.
In this section we report only on the differences between our current methodology
and the methodology in that work.

\subsection{Hybrid Advection Scheme}
\label{sec:hybrid}

It is well known that there is a trade-off between conservation of linear
momentum and conservation of angular momentum in grid-based, Eulerian codes
such as \castro. Although both quantities should be conserved analytically, we cannot
numerically guarantee that both hold. Consequently the choice of which
physical quantity to conserve must be motivated by analysis of the particular
problem to be done. In \cite{wdmergerI}, we demonstrated that numerical conservation
of linear momentum leads to white dwarf mergers that are spuriously caused by violation
of numerical conservation of angular momentum. Ideally our simulation code should be capable of evolving the
equations for both the linear momentum and the angular momentum, and be
able to switch between them as appropriate. Prior to this work, \castro\
already had the standard method implemented for evolving the linear momentum.
To solve the need for an angular momentum evolution, which could end up being more
appropriate for binary star systems than linear momentum evolution (a proposition we
will test in \autoref{sec:results}), we adopt the ``hybrid''
advection scheme presented by \cite{byerly:2014} (see also \cite{motl:2002} for
a similar equation set). Their approach is appropriate for physical situations
where there is a dominant angular momentum axis and it is desirable to conserve
that component as accurately as possible. This is certainly applicable for
studies of the stability of binary stellar systems. This section contains a
discussion of our implementation of that scheme in \castro.

For the sake of simplicity, let us assume that the rotation axis of the binary system is the $z$
axis and so the WDs orbit in the $xy$-plane. The core of the method is that instead
of solving the Euler equations for the linear momenta $(\rho u)$ and $(\rho v)$,
we instead solve the corresponding equations for the radial momentum
$s_R \equiv \rho v_R$ (where $v_R$ is the radial velocity with respect to the
rotation axis) and the angular momentum with respect to the $z$-axis,
$\ell_z \equiv R\rho v_\phi$ (where $v_\phi$ is the azimuthal velocity
and $R$ is the distance to the rotation axis).
The equation for the linear $z$-momentum is unchanged. We want to be able
to solve these equations while still operating in our preferred Cartesian
coordinate system. \cite{byerly:2014} present these equations as the following
(for the moment we neglect inclusion of external source terms such as gravity and rotation,
and we note that we have reorganized terms slightly relative to their presentation):
\begin{align}
  \frac{\partial(s_R)}{\partial t} &= -\nabla \cdot (s_R \mathbf{u})
    - \frac{1}{R}\left(x \frac{\partial p}{\partial x} + y \frac{\partial p}{\partial y}\right)
    + \frac{\ell_z^2}{\rho R^3} \label{eq:radial-momentum}\\
    \frac{\partial(\ell_z)}{\partial t} &= -\nabla \cdot (\ell_z \mathbf{u}) - \left(\frac{\partial (-py)}{\partial x} + \frac{\partial(px)}{\partial y}\right).\label{eq:angular-momentum}
\end{align}
Here $\mathbf{u}$ is the ordinary linear momentum used to advect any fluid quantity
on the grid, and $R = \sqrt{x^2 + y^2}$, where the coordinates $x$ and $y$ are defined
relative to the axis of rotation, and the origin is located on a zone corner
so that singularities are avoided. Note that $v_R = (1 / R)(x u + y v)$, and
$v_\phi = (1/R)(x v - y u)$. These equations can be straightforwardly derived from
the Euler equations for linear momentum by appropriate multiplication of $x$, $y$,
$x / R$, and $y/R$, and subsequent algebraic manipulation. The first term on the
right-hand-side is the standard advective flux term for any variable that advects with
the flow, and the state quantity inside this advective term can be derived on cell edges
using the primitive hydrodynamic variables after a Riemann solve has been performed.
The second term on the right-hand-side of \autoref{eq:angular-momentum} can be swept up
into the flux terms in the same way that the pressure term for the linear momenta
is often added to the fluxes, to make the conservation form explicit. The remainder
are the last two terms on the right-hand-side of \autoref{eq:radial-momentum}.
While maintaining second-order accuracy in time, the last term can be treated as a
cell-centered source term that we deal with using a predictor-corrector
method (similar to how we implement external forcing like gravity and rotation; see
\cite{castro} for details). The pressure gradients in the second-to-last term can be
constructed using the Riemann-solved edge states, and they are multiplied by zone
coordinates that are cell-centered. In principle this is second-order accurate in time
because the edge states are too; however, as a drawback, the pressure is unaware of the
source terms that are applied in the corrector step after a hydrodynamics update. The
benefit of this approach is that we do not need to construct a general cell-centered
gradient of the pressure using cell-centered state data, nor do we need to perform the
equation of state call that entails.

It is important to observe that in this formulation the radial momentum equation
is \textit{not} a conserved quantity: it should and will in general change with
time over the course of the simulation if the system is not in perfect rotational
equilibrium. The above form, absent of source terms, makes this clear. The angular
momentum term in \autoref{eq:radial-momentum} is effectively an outward centrifugal
force that tends to increase the radial momentum with time. This is a reflection of
the fact that mass on the domain with non-zero momentum tends to increase its radial
momentum with time. For example, consider a star initially on the $x$ axis moving
in the positive $y$ direction at constant speed: initially its radial momentum is
zero, but as its $y$ coordinate increases, more of its velocity becomes radial rather
than angular).  This term can thus also be thought of geometrically as expressing the
curved nature of the underlying coordinates for the radial momentum
\citep{motl:2002}. In the presence of a source term like gravity that can keep
such a star on an orbit at its original radius, the gravitational force acts as a
centripetal force that balances the centrifugal force, allowing the radial momentum
to remain constant with time, as it should when the orbital radius is constant.

In the absence of source terms, the angular momentum \textit{is} conserved
to machine precision (ignoring the effects of physical domain boundaries). In the
presence of source terms, the error in angular momentum conservation is of a
similar order of magnitude to the error in linear momentum conservation under
the influence of those source terms (see \autoref{eq:force-angular}). Since this
error is quite small for gravitational and rotational forces in practice, angular
momentum conservation is quite good, especially in comparison to the standard
method of evolving only the linear momentum. Thus this method is most appropriate
when the net radial momentum is small in comparison to the net angular momentum,
that is, when the motion is primarily azimuthal in nature.

To implement this method in \castro, we desired an approach that was flexible
and would leave as little imprint on the code structure as possible.
To this end, what we have done is to add three new ``hybrid'' momentum state
variables, corresponding to the radial momentum, angular momentum, and the linear
momentum component that is perpendicular to both. Whenever we update the normal
momentum state variables with hydrodynamics fluxes, we add the analogous fluxes
to the hybrid momenta --- that is, using the edge state values determined by the
final multi-dimensional Riemann solve, we construct the value of the hybrid
momenta on that zone edge, and then allow it to be transported as usual by the
advective velocity $\mathbf{u}$. When we update the normal momenta with an
external forcing $\mathbf{F}$ such as gravity, with $x$ and $y$
components $F_x$ and $F_y$, we apply the update to the hybrid momenta as well:
\begin{align}
  \left.\frac{\partial(s_R)}{\partial t}\right|_{F} &= -F_x \frac{x}{R} - F_y\frac{y}{R} \label{eq:force-radial}\\
  \left.\frac{\partial(\ell_z)}{\partial t}\right|_{F} &= F_x\, y - F_y\, x.\label{eq:force-angular}
  \end{align}

If in addition to evolving these variables we want to use them in place of the
standard linear momenta for determining the hydrodynamic flow, only one change
is needed for the standard timestep advance scheme: after source terms are
applied to the state, we recompute the normal momenta so that they are fully
consistent with the hybrid momenta. In practice for a \castro\ advance this
synchronization of the linear and hybrid momenta occurs twice: once after the
old-time source terms and the hydrodynamic update has been applied (these all
use time-level $n$ data to construct the update, which is expected to already have
synchronized linear and hybrid momenta), and once after new-time source
terms have been applied. The use of adaptive mesh refinement implies sychronization
in two other places: after a reflux that syncs up the fluxes between coarse and
fine grids; and, after interpolation has been performed to obtain fine grid data
from a coarser grid (since the spatially interpolated data may not respect the
consistency relationship between the two momenta). In this way, the linear momenta
are essentially placeholder variables that always reflect what the current radial
and angular momenta are, but it is quite useful to express the scheme in this way
because most of the \castro\ infrastructure is already set up to use the linear
momentum state variables for calculating source terms, so the hybrid scheme requires
minimal changes to the code architecture. For example, when we compute
the primitive variables from the conservative variables in preparation for an advective
hydrodynamic update, we will still be getting them from the linear momenta,
but these linear momenta are consistent with the conserved angular momentum,
so we get the same result as if we had modified the code to explicitly compute
the primitive variables as a function of the hybrid momenta. Consequently the
scheme presented here adds the ability to explicitly conserve angular momentum
in a simulation without a significant footprint on the simulation software.



\subsection{Rotation Revisited}\label{sec:rotation}

Another significant change we have made in \castro\ since \citet{wdmergerI} deals
with how rotation is applied as a source term. We noted in that paper that rotation
can be used to reduce advection errors in a binary system due to the fact that
material is moving less on the grid. However, we have also observed that rotation
source terms introduce the opportunity for other numerical problems to arise. In
particular, the most significant problem for us is that conservation of linear and
angular momentum is harder to accomplish with the standard rotation source terms.
Largely this is due to the fact that the rotation force is not expressed in a numerically
conservative fashion; the forces are evaluated at cell centers and do not inherently
have the telescoping property that allows for conservation of momentum in the case
of hydrodynamic fluxes. While gravity has the property that in practice momentum
conservation errors due to the gravitational force are typically quite small due to
an underlying telescoping behavior in the gravitational force (see the discussion
in Section 2.3.1 of \cite{wdmergerI}), this is harder to obtain for rotation (this is
easy to see when considering the Coriolis force, as the standard Coriolis forcing
involves cell-centered velocities that cannot trivially be evaluated in a way that
leads to numerical cancellation --- though see \citet{audusse:2009} for a method that
utilizes the edge-centered hydrodynamic fluxes to this end). A way around this issue
is to evolve variables measured in the \textit{inertial} frame, even when the reference
frame used for the advection is rotating. The benefits of this were observed by
\citet{kley:1998} in the context of protostellar accretion disk simulations (and the
method has since been used in a number of codes in that community, e.g. \citet{NIRVANA}
and \citet{FARGO3D}), and also by \citet{call:2010}, whose method was adopted by
\citet{byerly:2014} in the context of binary star simulations. The benefit of using
inertial frame variables in this context is that the rotational terms vanish as an explicit
source term to the angular momentum. \citet{kley:1998} points out that the rotational update
to the $z$-component of the angular momentum measured in the rotating frame (which we
temporarily call $\ell^\prime$) is:
\begin{equation}
  \left.\frac{\partial \ell}{\partial t}\right|^{\text{rotation}} = -2\rho\, v_r\, \omega\, R.\label{eq:rotating-angular-momentum}
\end{equation}
(Note that in \castro\ we construct all source terms, including rotation, for the linear
momenta and then apply the source term to the hybrid momenta using \autoref{eq:force-radial} and
\autoref{eq:force-angular}; \autoref{eq:rotating-angular-momentum} is automatically obtained this way.)
The term on the right-hand-side vanishes if instead we evolve the inertial frame $\ell$, and there
is no source term for the radial momentum either. If one is evolving the linear momenta measured
in the inertial frame, \cite{byerly:2014} give the form of the update:
\begin{equation}
  \left.\frac{\partial (\rho \mathbf{u})}{\partial t}\right|^{\text{rotation}} = -{\bm{\omega}} \times (\rho \mathbf{u})
\end{equation}
This is equal to one-half of the Coriolis force (with the centrifugal force not used; see
\cite{call:2010} for a further discussion of the meaning of this expression),
and is needed because the linear momentum components change when advected over a
rotating cylindrical grid. (Since the form of the update is similar to the Coriolis force,
we can still use the implicit update technique from Section 2.4 of \cite{wdmergerI} when
evaluating the new-time source term at time-level $n+1$.)

The ability to evolve state variables measured in the inertial frame when doing a
rotating frame simulation has been added as a runtime parameter in \castro. Note that
other than evolving inertial frame variables, other aspects of the rotating frame are
preserved --- for example, binary stars in a stable circular orbit will remain in the same
location on the grid as where they started, just as in a standard rotating frame simulation,
and the difference is mainly that the simulation velocity associated with those stationary
stars will read as the equivalent velocity of the star in the inertial frame. The only
meaningful wrinkle this introduces is that the hydrodynamic update needs to use the
rotating frame velocities, not the inertial frame velocities. So in the conversion from
conserved variables $\mathbf{U}$ to primitive variables $\mathbf{q}$, the latter being
the ones used by the hydrodynamics module to construct an advective update, we subtract
the rotating frame velocity from the conserved variables when constructing $\mathbf{q}$.
Then the rest of the hydrodynamics update proceeds as normal with rotating frame velocities.
As observed by \citet{byerly:2014}, this is acceptable because the divergence of the
rotational velocity field is zero, so it can legitimately be analytically subtracted
from the inertial velocity field. The payoff to the scheme is that we get the better
numerical behavior associated with the rotating frame velocities, as they are smaller
in magnitude with respect to the grid.



\subsection{Miscellaneous Other Changes}
\label{sec:hydro-other-changes}

- Density flux limiter



%==========================================================================
% Results
%==========================================================================
\section{Simulation Results}\label{sec:results}



%==========================================================================
% Conclusions
%==========================================================================
\section{Conclusions and Discussion}\label{sec:Conclusions and Discussion}


\acknowledgments

The authors thank Dominic Marcello and Zachary Byerly for helpful comments
regarding the hybrid advection scheme.

This research was supported by NSF award AST-1211563 and DOE/Office of
Nuclear Physics grant DE-FG02-87ER40317 to Stony Brook.  This
research used resources of the National Energy Research Scientific
Computing Center, which is supported by the Office of Science of the
U.S. Department of Energy under Contract No. DE-AC02-05CH11231.  An
award of computer time was provided by the Innovative and Novel
Computational Impact on Theory and Experiment (INCITE) program.  This
research used resources of the Oak Ridge Leadership Computing Facility
located in the Oak Ridge National Laboratory, which is supported by
the Office of Science of the Department of Energy under Contract
DE-AC05-00OR22725.

\clearpage

\bibliographystyle{../aasjournal}
\bibliography{../refs}


\clearpage
\appendix



\end{document}

