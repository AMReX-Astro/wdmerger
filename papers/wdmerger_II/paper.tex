\documentclass[twocolumn,numberedappendix]{../aastex6}

% these lines seem necessary for pdflatex to get the paper size right
\pdfpagewidth 8.5in
\pdfpageheight 11.0in

% for the red MarginPars
\usepackage{color}

% some extra math symbols
\usepackage{mathtools}

% allows Greek symbols to be bold
\usepackage{bm}

% allows us to force the location of a figure
\usepackage{float}

% allows comment sections
\usepackage{verbatim}

% Override choices in \autoref
\def\sectionautorefname{Section}
\def\subsectionautorefname{Section}
\def\subsubsectionautorefname{Section}

% MarginPars
\setlength{\marginparwidth}{0.75in}
\newcommand{\MarginPar}[1]{\marginpar{\vskip-\baselineskip\raggedright\tiny\sffamily\hrule\smallskip{\color{red}#1}\par\smallskip\hrule}}

\newcommand{\msolar}{\mathrm{M}_\odot}

% Software names
\newcommand{\boxlib}{\texttt{BoxLib}}
\newcommand{\castro}{\texttt{CASTRO}}
\newcommand{\wdmerger}{\texttt{wdmerger}}
\newcommand{\python}{\texttt{Python}}
\newcommand{\matplotlib}{\texttt{matplotlib}}
\newcommand{\yt}{\texttt{yt}}
\newcommand{\isoseven}{\texttt{iso7}}
\newcommand{\aproxthirteen}{\texttt{aprox13}}
\newcommand{\aproxnineteen}{\texttt{aprox19}}
\newcommand{\aproxtwentyone}{\texttt{aprox21}}

\begin{document}

%==========================================================================
% Title
%==========================================================================
\title{White Dwarf Mergers on Adaptive Meshes\\ II. Collisions}

\shorttitle{WD Mergers. II. Collisions}
\shortauthors{Katz et al. (2016)}

\author{Max P. Katz\altaffilmark{1}}
\author{Michael Zingale\altaffilmark{1}}
\author{Alan C. Calder\altaffilmark{1,2}}
\author{F. Douglas Swesty\altaffilmark{1}}
\author{Ann S. Almgren\altaffilmark{3}}
\author{Weiqun Zhang\altaffilmark{3}}

\altaffiltext{1}
{
  Department of Physics and Astronomy,
  Stony Brook University, Stony Brook, NY, 11794-3800, USA
}

\altaffiltext{2}
{
  Institute for Advanced Computational Sciences,
  Stony Brook University, Stony Brook, NY, 11794-5250, USA
}

\altaffiltext{3}
{
  Center for Computational Sciences and Engineering,
  Lawrence Berkeley National Laboratory, Berkeley, CA 94720
}

%==========================================================================
% Abstract
%==========================================================================
\begin{abstract}
We consider the collisions of white dwarfs as possible progenitors of Type Ia 
supernovae.

\end{abstract}
\keywords{supernovae: general - white dwarfs}

%==========================================================================
% Introduction
%==========================================================================
\section{Introduction}
\label{sec:introduction}

We consider the collisions of white dwarfs as possible progenitors of Type Ia 
supernovae, using the code \castro\ \cite{castro}.


%==========================================================================
% Numerical Implementation
%==========================================================================
\section{Numerical Methods}
\label{sec:numericalmethods}

The hydrodynamical equations solved in this paper were presented in \citet{wdmergerI}.
In this paper we focus our discussion on our implementation of the nuclear
reaction network and integrator.

%==========================================================================
% Problem Setup
%==========================================================================
\section{Problem Setup}
\label{sec:problemsetup}

We implement the white dwarf collision problem in \castro\ using our \wdmerger\
package. The white dwarf centers of mass are separated by a distance of four times
the (secondary) white dwarf radius. Their initial velocity is that of
two point-masses in free-fall towards each other, at the appropriate distance.


%==========================================================================
% 2D simulations
%==========================================================================
\section{Two-Dimensional Simulations}
\label{sec:2D}

Though the main focus of this work is fully three-dimensional simulations,
there is much to be learned from simpler, and much computationally cheaper,
two-dimensional axisymmetric collision simulations. These are implemented
in a cylindrical ($R-z$) coordinate system. We use these for three purposes.
First, in the regime where the 2D and 3D simulations can be usefully compared, 
zero-impact parameter simulations, to understand to what extent the
reduced-dimensionality approach can reproduce the 3D results and thus to
conclude whether 2D simulations can be used for obtaining scientific data
such as nucleosynthesis yields. Second, for testing the effect of various
code choices such as the number of isotopes in the nuclear network. A third
effect which comes for free with the others is that we can benchmark existing
and new code tools such as our nuclear postprocessing analysis. Prepared
with these choices and a better understanding of how reliable our approximations
are, we can then proceed to performing 3D simulations with our code choices
already made and with confidence in our analysis software.

\subsection{Dynamics}
\label{sec:2D:dynamics}

First we briefly consider whether the dynamical evolution of the system
is consistent with simple physics we understand. We ask two questions.
First, is the time it takes for the white dwarfs to collide consistent with
basic kinematics. Second, is the amount of energy released in the
collision by the nuclear reactions comparable to the amount of kinetic
energy we infer from observations of Type Ia supernovae?

\subsection{Timestep Limiting}
\label{sec:2D:timestep}

Here we discuss the effect of limiting the hydrodynamic timestep based on
changes in the internal energy in the nuclear reaction network.

\subsection{Nuclear Network}
\label{sec:2D:network}

Now we move to a discussion of the effect of including various isotopes
in the nuclear reaction network. In particular we compare the \aproxthirteen,
\aproxnineteen, and \aproxtwentyone reaction networks.

\subsection{Nucleosynthetic Postprocessing}
\label{sec:2D:postprocessing}

We consider also the results of our postprocessing step for nucleosynthesis
yields, to understand whether the results for the isotopes not included in
our networks make sense.



%==========================================================================
% 3D simulations
%==========================================================================
\section{Three-Dimensional Simulations}
\label{sec:3D}

Now that we have analyzed the various parts of our simulation software and
discussed our code choices, we consider full 3D simulations to understand
the potential for white dwarf collisions to be progenitors of Type Ia
supernovae. We focus on two main quantities of interest: the total amount of
nickel produced in the explosion, which directly powers the light curve
observed in SNe Ia; and, the gravitational wave signature produced by
these events.


%==========================================================================
% Conclusions
%==========================================================================
\section{Conclusions and Discussion}\label{Sec:Conclusions and Discussion}
\label{sec:conclusion}


\acknowledgments

This research was supported by NSF award AST-1211563 and DOE/Office of
Nuclear Physics grant DE-FG02-87ER40317 to Stony Brook. An award of
computer time was provided by the Innovative and Novel Computational
Impact on Theory and Experiment (INCITE) program.  This research used
resources of the Oak Ridge Leadership Computing Facility located in
the Oak Ridge National Laboratory, which is supported by the Office of
Science of the Department of Energy under Contract
DE-AC05-00OR22725. Project AST106 supported use of the ORNL/Titan
resource.  This research used resources of the National Energy
Research Scientific Computing Center, which is supported by the Office
of Science of the U.S. Department of Energy under Contract
No. DE-AC02-05CH11231.  Results in this paper were obtained using the
high-performance LIred computing system at the Institute for Advanced
Computational Science at Stony Brook University, which was obtained
through the Empire State Development grant NYS \#28451.

This research has made use of NASA's Astrophysics Data System 
Bibliographic Services. In addition, this research has made use
of the AstroBetter blog and wiki.

\clearpage

\bibliographystyle{../aasjournal}
\bibliography{../refs}


\clearpage
\appendix



\end{document}

