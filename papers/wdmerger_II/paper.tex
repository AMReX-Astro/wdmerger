\documentclass[twocolumn,numberedappendix]{../aastex6}

% these lines seem necessary for pdflatex to get the paper size right
\pdfpagewidth 8.5in
\pdfpageheight 11.0in

% for the red MarginPars
\usepackage{color}

% some extra math symbols
\usepackage{mathtools}

% allows Greek symbols to be bold
\usepackage{bm}

% allows us to force the location of a figure
\usepackage{float}

% allows comment sections
\usepackage{verbatim}

% Override choices in \autoref
\def\sectionautorefname{Section}
\def\subsectionautorefname{Section}
\def\subsubsectionautorefname{Section}

% MarginPars
\setlength{\marginparwidth}{0.75in}
\newcommand{\MarginPar}[1]{\marginpar{\vskip-\baselineskip\raggedright\tiny\sffamily\hrule\smallskip{\color{red}#1}\par\smallskip\hrule}}

\newcommand{\msolar}{\mathrm{M}_\odot}

% Software names
\newcommand{\boxlib}{\texttt{BoxLib}}
\newcommand{\castro}{\texttt{CASTRO}}
\newcommand{\microphysics}{\texttt{Microphysics}}
\newcommand{\wdmerger}{\texttt{wdmerger}}
\newcommand{\python}{\texttt{Python}}
\newcommand{\matplotlib}{\texttt{matplotlib}}
\newcommand{\yt}{\texttt{yt}}
\newcommand{\vode}{\texttt{VODE}}
\newcommand{\isoseven}{\texttt{iso7}}
\newcommand{\aproxthirteen}{\texttt{aprox13}}
\newcommand{\aproxnineteen}{\texttt{aprox19}}
\newcommand{\aproxtwentyone}{\texttt{aprox21}}

\begin{document}

%==========================================================================
% Title
%==========================================================================
\title{White Dwarf Mergers on Adaptive Meshes\\ II. Collisions}

\shorttitle{WD Mergers. II. Collisions}
\shortauthors{Katz et al. (2016)}

\author{Max P. Katz\altaffilmark{1}}
\author{Michael Zingale\altaffilmark{1}}
\author{Alan C. Calder\altaffilmark{1,2}}
\author{F. Douglas Swesty\altaffilmark{1}}
\author{Ann S. Almgren\altaffilmark{3}}
\author{Weiqun Zhang\altaffilmark{3}}

\altaffiltext{1}
{
  Department of Physics and Astronomy,
  Stony Brook University, Stony Brook, NY, 11794-3800, USA
}

\altaffiltext{2}
{
  Institute for Advanced Computational Sciences,
  Stony Brook University, Stony Brook, NY, 11794-5250, USA
}

\altaffiltext{3}
{
  Center for Computational Sciences and Engineering,
  Lawrence Berkeley National Laboratory, Berkeley, CA 94720
}

%==========================================================================
% Abstract
%==========================================================================
\begin{abstract}
We consider the collisions of white dwarfs as possible progenitors of Type Ia 
supernovae.

\end{abstract}
\keywords{supernovae: general - white dwarfs}

%==========================================================================
% Introduction
%==========================================================================
\section{Introduction}
\label{sec:introduction}

We consider the collisions of white dwarfs as possible progenitors of Type Ia 
supernovae, using the code \castro\ \cite{castro}.


%==========================================================================
% Numerical Implementation
%==========================================================================
\section{Numerical Methods}
\label{sec:numericalmethods}

The hydrodynamical equations solved in this paper were presented in \citet{wdmergerI}.
In this paper we focus our discussion on our implementation of the nuclear
reaction network and integrator. There are three main areas of concern: first,
specifying what nuclides are in the network and what reaction rates link the
various nuclides; second, once the system of ODEs governing the evolution of
the nuclear species has been written down, how to append to them ODEs governing
the evolution of the thermodynamics, in particular the temperature; third,
coupling the nuclear reaction updates to the evolution of the hydrodynamical
system. We will deal with each of these in turn in the following subsections.

\subsection{Nuclear Network}
\label{sec:network}

White dwarfs are mainly composed of $\alpha$-chain particles, primarily ${}^4$He,
${}^{12}$C, ${}^{16}$O, ${}^{20}$Ne, and ${}^{24}$ Mg. Therefore the core of
any network appropriate for modeling nuclear burning in white dwarfs will be
these alpha chain nuclides, with the idea being that links up the $\alpha$-chain
will eventually get us to ${}^{56}$Ni, the nuclide responsible for the
energy output of Type Ia supernovae. In this paper we consider four networks
to do this, presented in order of increasing complexity. The most simple is
\isoseven\ \citep{timmes:2000}, which includes all of the aforementioned isotopes and
${}^{28}$Si (see also \citet{hix:1998}). ${}^{28}$Si effectively measures the
equilibrium state of silicon-group elements, and ${}^{56}$Ni effectively measures
the equilibrium state of iron-group elements, with the link between them governed
by the effective loss or gain of seven $\alpha$-particles. This type of network
was used by \citet{rosswog:2009} for their collision calculations in SPH.

Next is \aproxthirteen\ \citep{timmes:1999,timmes:2000}. This includes
all of the isotopes of \isoseven, and all of the $alpha$-chain particles between
silicon and nickel (${}^{32}$S, ${}^{36}$Ar, ${}^{40}$Ca, ${}^{44}$Ti, ${}^{48}$Cr,
and ${}^{52}$Fe). This network was used by \citet{hawley:2012} and \citet{raskin:2010}.
\citet{loren-aguilar:2010} and \citet{garcia-senz:2013} used a very similar network
that included additionally ${}^{60}$Zn. The \aproxnineteen\ network \citep{timmes:1999}
builds on \aproxthirteen\ by including isotopes for hydrogen burning and explicit
tracking of photodisintegration into ${}^{54}$Fe. This network was used by
\citet{kushnir:2013}, \citet{kushnir:2014}, \citet{rosswog:2009} for their
calculations with FLASH, and \citet{papish:2015}. Finally we will also
consider \aproxtwentyone, which includes all of the above plus ${}^{56}$Cr
and ${}^{56}$Fe and related reaction links. The primary virtue of using
the latter two networks is that they allow us to track changes away from
$Y_e = 0.5$.

All four of these networks have been ported into a form that is consistent
with the \boxlib\ codes, in the freely available \microphysics\ code
repository\footnote{\microphysics\ can be obtained at \url{https://github.com/BoxLib-Codes/Microphysics}.},
a collection of microphysical routines that are designed to be used in our
hydrodynamics codes. These can be easily swapped at compile time by using the 
appropriate makefile variable.

\subsection{Nuclear Burning}
\label{sec:burner}

Given a set of nuclides and the reaction links between them, we now consider
how a burning step is performed in our software. The integration of the stiff,
implicit system is performed by \vode\ \citep{vode}, a copy of which is
provided with \castro. To \vode\ we provide an integration state containing
the molar abundances $Y_{n} = X_{n} / A_{n}$, where $X$ is the mass fraction
of the abundance in the zone and $A$ is the number of nucleons. The integration
of
\begin{equation}
  \frac{d\bm{Y}}{dt} = f(\mathbf{Y})
\end{equation}
is determined using a right-hand-side provided by the nuclear burning network.
When it comes to the thermodynamical evolution, the internal energy $e$ of the zone
will change when the nuclear abundances evolve. During the integration we track
the energy release from the system using
\begin{equation}
  \frac{\partial e}{\partial t} = N_A \sum_{n} \frac{\partial Y_{n}}{\partial t} m_{n} c^2,
\end{equation}
where $c$ is the speed of light and $m_n$ is the mass of each nuclide.

In a hydrostatic burn, we keep $\rho$ and $T$ fixed throughout, and use
the energy released at the end to compute a final temperature that is
thermodynamically consistent with the new internal energy. By contrast,
in a self-heating burn, we allow the temperature to evolve in response
to the burning (see \citet{maestro3}):
\begin{equation}
  \frac{dT}{dt} = \frac{1}{c_V}\left(\frac{\partial e}{\partial t} - \sum_n \frac{\partial e}{\partial X_n}\frac{\partial X_n}{\partial t}\right)
\end{equation}
Here $c_V$ is the specific heat at constant volume, and $(\partial e/\partial X_n)$
is the derivative of the internal energy with respect to the mass fraction of a given
species (with other thermodynamic variables held constant). Both quantities are provided by the
equation of state. During this burn, we can keep these thermodynamic
derivatives constant using their value at the beginning of the run, or at each step we
can choose to re-evaluate the equation of state using the latest value of $(\rho, T)$.
The latter is more expensive but also more accurate, and we use it in this paper.
A third option presented by \citet{raskin:2010} is a so-called ``hybrid'' mode.
In this mode, by default we do a hydrostatic burn. If that burn fails, or if the net
energy change is negative, we do the burn again in self-heating mode. All three options
are implemented in our burner software.

\subsection{Coupling to the Hydrodynamics}
\label{sec:hydrocoupling}

In \castro, the reactions are coupled to the hydrodynamics using Strang splitting.
In a given timestep advance $\Delta t$, we first evolve the reactions alone through
a time interval $\Delta t / 2$. Then, we evolve the hydrodynamics for $\Delta t$,
and we evolve the reactions again for a further $\Delta t / 2$. The principal
drawback of this approach is that the reactions and the hydrodynamics can become
decoupled from each other. A common solution to this problem presented in
the literature has been to limit the size of the timestep and thereby limit the
extent of this decoupling \citep{raskin:2010,hawley:2012}, which we adopt here 
and have implemented in \castro. Defining the nuclear energy injection timescale 
$t_e$:
\begin{equation}
  t_e \equiv \frac{e}{|\dot{e}|},
\end{equation}
where $\dot{e}$ is an estimate of the time rate of change of the internal energy
from nuclear burning, we define a burning-limited timestep $\Delta t_{b}$:
\begin{equation}
  \Delta t_{b} = f_{\Delta t}\, t_e.
\end{equation}
We estimate $\dot{e}$ using the energy change in the last burning
update, divided by the length of the burning timestep. The factor $f_{\Delta t}$ then
determines by what fraction we would like to allow the internal energy to change
in the current timestep (assuming that $\dot{e}$ does not change quickly from
timestep to timestep, which is not always the case). By making $f_{\Delta t}$ smaller,
we can control the magnitude of the decoupling between the reactions and the hydro.
We choose by default $f_{\Delta t} = 0.1$, a choice which is on par with others in 
the literature. The sensitivity of results to this choice will be discussed in 
\autoref{sec:2D:timestep}. The factor $f_{\Delta t}$ can be set at runtime in \castro.

At the start of each advance, we limit the size of the timestep to be the smaller
of the minimum hydro timestep (limited by the CFL condition), and the minimum of the
burning timestep across all zones.

\citet{kushnir:2013} point out this is not the only way for the burning to
become numerically unstable; another failure mode is when the energy injection timescale
$t_e$ is shorter than the sound-crossing time $t_s$ in a zone.
The goal is to ensure that the following condition holds:
\begin{equation}
  t_s \leq f_{\Delta x} t_e \label{eq:burning_limiter_2}
\end{equation}
The sound crossing time, $t_s$, is given by $\Delta x / c_s$, 
where $c_s$ is the sound speed and $\Delta x$ is the (minimum) 
zone width. The parameter $f_{\Delta x}$ then determines the minimum
ratio of the nuclear energy generation timescale to the 
sound-crossing time. \citet{kushnir:2013} choose $f_{\Delta x} = 0.1$ 
for their simulations, and we do too (this parameter can be set 
at runtime in \castro).

\citet{kushnir:2013} implemented this criterion by artificially 
limiting the magnitude of the energy release after a burn. As 
this is not physical, we take a different approach. Since we 
cannot directly control the energy injection timescale, we 
must find a way to alter the sound-crossing timescale. 
We can achieve this by adding levels of refinement in 
regions that do not satisfy \autoref{eq:burning_limiter_2},
which effectively lowers the $\Delta x$ and thus the
sound-crossing time. We keep tagging zones for refinement
based on this criterion until the criterion is satisfied
on the finest level. Since the concern is regions that 
may detonate, we also tag nearby zones in a buffer region
which do not themselves satisfy the criterion,
so that a detonation in a single timestep cannot 
escape into non-refined regions. The width of the buffer 
region should thus be at least as large as the number of 
timesteps before a regridding procedure is performed.
We choose a value of two for both the number of zones in the 
buffer region and the number of steps in between regrids,
for all simulations in this paper.

While this approach may add several AMR levels 
to a simulation, we agree with \citet{kushnir:2013} that 
solving this numerical instability is crucial to avoiding
early, unphysical detonations, and so it is unavoidable
if a correct evaluation of the burning phase is desired.
A simulation that does not solve this problem in this way
(or some analogous manner) will not obtain the correct amount 
of burning, and will not converge properly with resolution. 
Fortunately, the regions where this criterion is not satisfied 
are fairly localized on the grid, so the amount of additional 
computational work is not exceedingly large.




%==========================================================================
% Problem Setup
%==========================================================================
\section{Problem Setup}
\label{sec:problemsetup}

We implement the white dwarf collision problem in \castro\ using our \wdmerger\
package. The white dwarf centers of mass are separated by a distance of four times
the (secondary) white dwarf radius. Their initial velocity is that of
two point-masses in free-fall towards each other, at the appropriate distance.


%==========================================================================
% 2D simulations
%==========================================================================
\section{Two-Dimensional Simulations}
\label{sec:2D}

Though the main focus of this work is fully three-dimensional simulations,
there is much to be learned from simpler, and much computationally cheaper,
two-dimensional axisymmetric collision simulations. These are implemented
in a cylindrical ($R-z$) coordinate system. We use these for three purposes.
First, in the regime where the 2D and 3D simulations can be usefully compared, 
zero-impact parameter simulations, to understand to what extent the
reduced-dimensionality approach can reproduce the 3D results and thus to
conclude whether 2D simulations can be used for obtaining scientific data
such as nucleosynthesis yields. Second, for testing the effect of various
code choices such as the number of isotopes in the nuclear network. A third
effect which comes for free with the others is that we can benchmark existing
and new code tools such as our nuclear postprocessing analysis. Prepared
with these choices and a better understanding of how reliable our approximations
are, we can then proceed to performing 3D simulations with our code choices
already made and with confidence in our analysis software.

\subsection{Dynamics}
\label{sec:2D:dynamics}

First we briefly consider whether the dynamical evolution of the system
is consistent with simple physics we understand. We ask two questions.
First, is the time it takes for the white dwarfs to collide consistent with
basic kinematics. Second, is the amount of energy released in the
collision by the nuclear reactions comparable to the amount of kinetic
energy we infer from observations of Type Ia supernovae?

\subsection{Timestep Limiting}
\label{sec:2D:timestep}

Here we discuss the effect of limiting the hydrodynamic timestep based on
changes in the internal energy in the nuclear reaction network.

\subsection{Nuclear Network}
\label{sec:2D:network}

Now we move to a discussion of the effect of including various isotopes
in the nuclear reaction network. In particular we compare the \aproxthirteen,
\aproxnineteen, and \aproxtwentyone reaction networks.

\subsection{Nucleosynthetic Postprocessing}
\label{sec:2D:postprocessing}

We consider also the results of our postprocessing step for nucleosynthesis
yields, to understand whether the results for the isotopes not included in
our networks make sense.



%==========================================================================
% 3D simulations
%==========================================================================
\section{Three-Dimensional Simulations}
\label{sec:3D}

Now that we have analyzed the various parts of our simulation software and
discussed our code choices, we consider full 3D simulations to understand
the potential for white dwarf collisions to be progenitors of Type Ia
supernovae. We focus on two main quantities of interest: the total amount of
nickel produced in the explosion, which directly powers the light curve
observed in SNe Ia; and, the gravitational wave signature produced by
these events.


%==========================================================================
% Conclusions
%==========================================================================
\section{Conclusions and Discussion}\label{Sec:Conclusions and Discussion}
\label{sec:conclusion}


\acknowledgments

This research was supported by NSF award AST-1211563 and DOE/Office of
Nuclear Physics grant DE-FG02-87ER40317 to Stony Brook. An award of
computer time was provided by the Innovative and Novel Computational
Impact on Theory and Experiment (INCITE) program.  This research used
resources of the Oak Ridge Leadership Computing Facility located in
the Oak Ridge National Laboratory, which is supported by the Office of
Science of the Department of Energy under Contract
DE-AC05-00OR22725. Project AST106 supported use of the ORNL/Titan
resource.  This research used resources of the National Energy
Research Scientific Computing Center, which is supported by the Office
of Science of the U.S. Department of Energy under Contract
No. DE-AC02-05CH11231.  Results in this paper were obtained using the
high-performance LIred computing system at the Institute for Advanced
Computational Science at Stony Brook University, which was obtained
through the Empire State Development grant NYS \#28451.

This research has made use of NASA's Astrophysics Data System 
Bibliographic Services. In addition, this research has made use
of the AstroBetter blog and wiki.

\clearpage

\bibliographystyle{../aasjournal}
\bibliography{../refs}


\clearpage
\appendix



\end{document}

